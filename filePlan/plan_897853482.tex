\documentclass[tikz,border=3.14mm]{standalone}
\usepackage{tikz}

\begin{document}

\begin{tikzpicture}

% Координаты четырехугольника
\coordinate (A) at (0, 0);
\coordinate (B) at (2, 3);
\coordinate (C) at (4, 0);
\coordinate (D) at (3.5, -3);

% Диагонали четырехугольника пересекаются в точке E
\coordinate (E) at (intersection of A--C and B--D);

% Постройка описанной окружности треугольника ADE
\draw[dashed] (A) -- (D) -- (E) -- cycle;
\draw[name path=circumcircleADE] (A) circle through (E);

% Определение точек P и Q
\path[name path=AB] (A) -- (B);
\path[name path=CD] (C) -- (D);
\path[name intersections={of=circumcircleADE and AB, by=P}];
\path[name intersections={of=circumcircleADE and CD, by=Q}];

% Рисунок четырехугольника и диагоналей
\draw (A) -- (B) -- (C) -- (D) -- cycle;
\draw (A) -- (C);
\draw (B) -- (D);

% Окружность
\draw[name path=circumcircleADE] (A) circle through (D);

% Рисуем точки A, B, C, D, E, P, Q
\fill[black] (A) circle (1.5pt) node[below left] {A};
\fill[black] (B) circle (1.5pt) node[above left] {B};
\fill[black] (C) circle (1.5pt) node[below right] {C};
\fill[black] (D) circle (1.5pt) node[below right] {D};
\fill[black] (E) circle (1.5pt) node[right] {E};
\fill[black] (P) circle (1.5pt) node[above left] {P};
\fill[black] (Q) circle (1.5pt) node[below] {Q};

\end{tikzpicture}

\end{document}
